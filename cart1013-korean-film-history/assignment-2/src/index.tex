\maketitle

\section*{<박하사탕>(1999)}

% 시대가 한 사람을 어떻게 망가지게 만들었는지 고발하는 영화.
<박하사탕>은 영호가 걸어온 삶의 궤적을 차근차근 되짚으며 시대가 어떻게 영호를 괴물로 만들었는지 고발한다. 영화는 하필 2000년 1월 1일에 개봉했다. 21세기를 여는 새해 첫날, 새천년의 희망을 노래하는 대신 지난 20세기의 비극적인 역사를 조명했다. `\oilpal 민주화운동 등에 관한 특별법'은 1995년에 제정되었고, 신군부의 군사반란과 내란행위는 1997년에야 판결이 확정됐다. 영호는 자신의 인생을 망쳐놓은 딱 한 놈을 죽이겠다며 마지막 재산을 털어 권총을 산다. 증권사 직원, 사채업자, 동업자 친구까지, 영호가 생각하기엔 자기 인생을 이렇게 망쳐놓은 사람이 너무 많다. 하지만 영화는 그의 인생을 망친 사람을 정확히 지목하고 있다. 1980년 5월 영호를 광주로 보낸 그자를.

% 전두환, 노태우 판결 확정(1997), 5.18민주화운동 등에 관한 특별법 제정(1995)
% 수많은 김영호를 만들어낸 그들은 김영호와 같은 최후를 맞지 않았다.
6$\cdot$29 선언과 함께 전두환이 물러났음에도 직선제 선거에서 노태우가 당선되며 군정종식은 이뤄지지 못했다. 1988년 제5공화국 청문회는 신군부의 핵심에 책임을 묻지 못했다. 결국 영호를 광주로 보낸 신군부 수뇌부가 처벌받기까지는 6$\cdot$29 선언으로부터 10여 년이 걸렸다. 그런데 전두환과 노태우가 형 집행을 확정받기도 전부터 두 사람에 대한 사면이 논의됐다. 제15대 대통령 선거에 출마한 후보들은 앞다투어 전두환, 노태우의 사면복권을 공약했고, 문민정부는 국민 대통합을 이유로 전두환과 노태우를 사면했다. 군사정권의 직접적인 피해자라는 당사자성을 가진 김영삼 전 대통령과 김대중 전 대통령은 대승적인 차원에서 신군부를 용서했지만, 한국은 부당한 권력을 청산하는 역사적 경험을 얻지 못했다. 역사적 경험의 부재는 수십 년에 걸쳐 청구서를 보내왔다. 2010년대 후반까지도 정치인에게 \oilpal이 민주화운동인지, 폭동인지 묻는 질문이 이상하지 않았다. \oilpal 북한개입설은 역사에 다양한 해석이 있을 수 있다는 논리로 보호받았다. 2020년 미래통합당 당 대표가 그간의 `\oilpal 망언'을 사과하고 정강정책에 `\oilpal 정신'을 담기까지 한국의 보수세력은 \oilpal을 민주화운동으로 인정하지 않았다. 영호는 끝까지 진정 자신의 삶을 망친 자들에게 책임을 묻지 못했고, 수많은 영호를 양산한 전두환은 안락한 말년을 보냈다. 영호의 비극은 영호만의 비극이 아니라 한국 현대사의 비극이다.

% 근대의 폭력성
<박하사탕>이 영호의 삶을 되돌아보며 서사를 이끌어가는 핵심적인 메타포는 기차다. <열차의 도착>(1895)부터 기차는 영화의 역사와 함께했다. 굉음을 내며 빠르게 달리는 거대한 기차는 근대의 욕망과 기술을 넘어, 근대 그 자체를 상징해 왔다. 아래로부터의 혁명으로 구축된 서구의 근대와 달리, 아시아에서는 위로부터의 개혁으로 근대가 도입되었다. 한반도에서 근대의 역사는 폭력의 역사이기도 하다. 일제는 수탈을 위해 한반도를 종단하는 철도를 부설하면서 한반도에 일방적으로 근대를 이식했다. 해방 이후 한반도의 종단 철도는 절단됐고, 미군정은 남한의 철도를 국유화하며 다시 한번 한반도에서 근대를 재정립했다. 무궁화호, 새마을호, 비둘기호를 연이어 개통한 1960-70년대 국가의 숙원은 `조국근대화'였다. ``싸우면서 일하고 일하면서 싸우자''는 박정희의 권위주의적 근대화는 결국 반공 국민국가를 만들기 위한 도구였다. 1980년대 신군부가 추구한 근대화 역시 그전과 다를 바 없이 신속하고 폭력적이었다. 그래서 <박하사탕>에서 철로를 반대로 달리는 기차는 한국의 근현대사를 거꾸로 달리는 기차이자, 폭력의 역사를 되돌아보는 과정이라고 할 수 있을 것이다.

% 영호를 연민하는가?
기차가 1980년 5월에 이르기 전까지 영호는 철저히 비난의 대상으로 그려진다. 야유회에 나타나 난동을 부리고, 아내 홍자의 외도에 분노한 뒤 자신도 외도를 한다. 운동권 학생을 고문하고, 순임 앞에서 홍자를 추행한다. 영호가 반복해서 말하는 과거의 첫사랑은 그저 위선으로 보인다. 하지만 영호가 광주에서 겪은 충격적인 사건으로 그 순수함을 잃게 되었다는 사실이 밝혀질 때 영호의 개인사는 현대사로 확장되고, 관객은 끝내 시대의 소모품으로서 영호를 바라보게 된다. 이창동 감독은 시간을 역행하는 서사구조를 통해 관객으로 하여금 영호를 연민하도록 만들었다. 이러한 접근은 자칫 잘못하면 영호에게 면죄부를 주는 것처럼 비칠 수 있다. <박하사탕>이 영호의 폭력성을 온전히 외부의 책임으로 돌린다는 지적도 있다.\footnote{김혜남, \snm{영화 `박하사탕'에 대한 정신분석적 접근}, \bnm{정신분석} 제11권 제1호, 한국정신분석학회, 2000, 77쪽.} 실제로 계엄군으로 투입된 군인과 고문을 자행한 경찰 중 과거를 증언하고 사죄한 이들은 많지 않으며, 이들은 아직도 신군부를 옹호하고, \oilpal 민주화운동이 폭동이라고 주장한다. 그럼에도 불구하고, 나는 <박하사탕>이 폭력의 역사를 영호 개인이 지닌 본성으로 환원하는 대신, 그를 둘러싼 환경과 시대에 초점을 맞췄다는 점에 의미가 있다고 생각한다. 한 인간이 범죄자가 되는 사회구조적인 원인과 과정을 분석하는 것에 비해 개인에게 책임을 묻고, 개인을 단죄하는 것은 비교적 쉽다. 하지만 그 오랜 시간, 그 많은 이들이 국가폭력에 가담했다는 사실은 분명 개개인의 의지와 선택 이전에 구조적인 원인이 작용했음을 방증한다.

한나 아렌트는 역사적 악행이 악의를 가진 자들에 의해 행해졌다기보다는, 일상적인 업무를 수행한 평범한 이들에 의해 행해졌다는 `악의 평범성'을 주장했다. 1970년대 유신체제 하에 10대를 보내고, 20대 초반에 계엄군이 된 김영호는 그의 삶에서 복종과 폭력을 내면화했을 것이다.\footnote{최치원, \snm{\oilpal 민주화운동에 대한 하나의 실존적 해석: `어두운 시대'에 감추어진 `악의 평범성' 문제}, \bnm{민주주의와 인권} 제11권 제3호, 전남대학교 5.18연구소, 2011, 23쪽} 영호에게 내재된 본성이 폭력적이었든, 그렇지 않았든 그가 악을 행하도록 방관, 권장한 주체는 국가권력과 제도, 그리고 부당한 권력을 청산하지 못한 한국의 역사였다. 이렇게 <박하사탕>은 거대한 시대적 조류와 구조 속에서 개인이 얼마나 나약한 존재이고, 얼마나 쉽게 타락할 수 있는지 보여준다. 그러나 <박하사탕>의 성찰이 무색하게, 김영호가 기차 앞에 몸을 내던진 뒤에도 용산참사, 쌍용차 사태, 백남기 농민 사망에 이르기까지 수차례 이어진 국가폭력의 현장에서 우리 현대사는 또다시 수많은 김영호와 그로 인한 피해자를 낳고 말았다.

\section*{<마더>(2009)}

% 모성 신화. 많은 예술 작품이 모성을 신성시하고 숭배해왔음.
% 모성은 가부장적 질서를 유지하기 위한 도구로 사용되어 옴.
모성애는 신성한 것으로 여겨져 왔다. 어머니는 위대하며, 그 어떤 사랑도 어머니의 사랑을 뛰어넘지 못한다. 그런 사랑을 베풀지 않는 여성은 어머니로서, 그리고 여성으로서 자격을 상실한 것으로 평가받는다. <마더>는 이러한 모성신화를 비틀어 과잉된 모성을 공포스러운 모습으로 묘사한다. 봉준호 감독의 다른 영화 <괴물>(2006), <설국열차>(2013), <기생충>(2019) 등과 달리 <마더>에는 사회 문제에 대한 감독의 메시지가 직접적으로 드러나지 않는다. 오히려 개인의 심리에 집중하는 이야기 같아 보이기도 하고, 잘 만들어진 장르 영화처럼 보이기도 한다. 하지만 그 이면에서 <마더>는 한국 사회가 정의한 모성을 향해 의문을 던진다.

% 학습된 모성, 가부장 이데올로기
사전적 의미에서의 모성은 어머니 역할을 하는 생물 개체가 보이는 객관적 특성을 의미할 뿐이다. 새끼를 보호하고 양육하는 행위뿐만 아니라, 뻐꾸기가 다른 새의 둥지에 알을 낳는 것도 암컷 뻐꾸기의 모성이다. 그러나 자연 그대로의 성질을 의미하는 모성과 달리, 사회는 여성에게 `어머니다움'으로서의 모성을 요구한다. 문화적으로 규정된 모성은 아이에게 무한한 사랑과 맹목적인 헌신을 쏟는 모성애가 모든 여성이 가진 본능이며, 그런 모성애를 실천하는 것이 바람직한 어머니상이라는 모성신화로 거듭난다. 그러나 모성애는 본능이 아닌, 학습된 문화라는 것이 이미 밝혀져 있다. 모성애가 문화라는 것은 오늘날 흔히 생각하는 `어머니다움'이 시대, 계층, 성적 지향에 따라 다르게 나타난다는 의미다. 가령 중세 이후 유럽 중상류층 가정의 경우, 영유아는 시골의 유모에게, 아동은 가정교사에게, 청소년은 기숙학교에게 위탁하는 것이 일반적이었다. 이들에게는 일반적으로 오늘날과 같은 모성애가 요구되지 않았으나, 남성 지식인들은 종교적인 관점에서 문제를 제기했다. 이들은 신이 여성에게 유방을 만들어 준 이유는 아이에게 젖을 물리기 위함이며, 모든 여성은 신이 부여한 모성의 의무를 이행해야 한다고 주장했다.\footnote{세라 블래퍼 허디, ``어머니의 탄생: 모성, 여성, 그리고 가족의 기원과 진화'', 사이언스북스, 황희선 역, 2010, 41쪽.} 오늘날의 모성신화는 이와 같은 계몽 운동을 기반으로, 산업혁명 이후 가족이 하나의 생산단위로 다뤄지며 형성된 것이다. 성별과 관계없이 모두가 생산에 참여한 전근대 사회와 달리 근대 도시 사회에는 공장에서 노동을 하는 남성 노동자와 집에서 가사를 담당하는 여성 주부로 성별분업 구조가 자리 잡았다. 이러한 분업 구조는 여성의 모성을 강조함으로써 고도화되었고, 전쟁으로 인해 여성이 공장 노동에 참여하게 된 이후에도 이어지면서 여성의 사회 진출을 방해하는 요소가 되었다. 가부장적 이데올로기가 지배하는 사회에서 모성은 어떤 여성이 올바른 여성인지, 어떤 여성이 그릇된 여성인지 평가하는 기준이 된다. 심지어는 여성 자신으로 하여금 스스로 모성의 기준에 부합하는지 끊임없이 확인하게 만든다. 이렇게 총체적인 모성신화, 모성을 향한 숭배와 모성애의 신비화는 오랜 시간 가부장적 이데올로기를 공고히 해왔다.

% 김혜자 배우 클리셰.
<전원일기>(1980-2002) 등의 작품으로 `국민엄마' 별명을 얻은 김혜자 배우가 <마더>에서 어머니 역으로 분한 것은 어떻게 보면 클리셰적이다. 그런데 만인에게 자애로운 `국민엄마'가 걷잡을 수 없이 기괴해지는 모습을 보이며 김혜자 배우의 클리셰는 반전하게 된다. 여기서 주목할 만한 부분은 <마더>의 어머니가 모성신화에 반하는 행동을 함으로써 공포를 자아내는 것이 아니라, 오히려 모성애를 충실히 실천함으로써 공포를 자아낸다는 점이다. 어머니는 도준의 결백을 밝히기 위해 고문과 살인까지 불사한다. 이렇게 과장된 모성애의 실천은 우리 사회가 숭배해온 모성신화를 향해 `정말 모성이 아름답기만 한가'라는 의문을 제기한다. <마더>는 가부장제가 납작하게 정의해온 모성을 뒤엎으며 강렬한 방식으로 모성의 다면성을 보여준다. 어머니는 외아들 도준을 지키기 위해 한시도 눈을 떼지 못한다. 그러나 ``한 대 치면 두 대 깐다. 무시하면 작살낸다.''라는 어머니의 가르침은 결국 살인이라는 결과를 낳고 말았다. 한편 어머니의 영향에서 벗어난 도준은 오히려 성숙해진다. 구치소에서 나온 도준은 제대로 젓가락질을 하고, 어머니에게 물을 떠주고, 화재 잔해에서 주운 어머니의 침통을 전달하며 이런 걸 흘리고 다니면 어떡하냐는 핀잔을 주기까지 한다. <마더>에서 자식을 위해 무한히 헌신하는 어머니의 모성애는 결과적으로 자식을 파멸시키고, 어머니 스스로까지 파멸시켰다. 도준이 구치소에서 어머니가 자신에게 농약을 먹였던 과거를 기억해내자, 어머니는 엄마가 얼마나 힘들었으면 그랬겠냐며 항변한다. 그런 어머니의 모습에서 어머니를 구속하고 있는 모성신화가 어쩌면 어머니의 자아실현 수단으로 작용하지는 않았을까 생각한다.

<마더>의 세계는 남성의 세계다. 세계를 차지한 남성들은 어머니를, 그리고 아정을 착취한다. 부성애와 비교해 지나칠 정도로 모성애를 강조하는 현실 세계를 반영하듯, 그런 남성의 세계에 아버지로서의 남성은 등장하지 않는다. 어머니로서, 그리고 여성으로서 <마더>의 어머니는 남성의 세계에 저항하는 것이 아니라, 그 질서를 노련하게 따르면서 자신의 목적을 달성했다. 이로 인해 나쁜 일, 끔찍한 기억을 얻는 어머니는 기억을 잊게 해준다는 침 자리에 스스로 침을 놓는다. 도준의 기억을 억제해 온 침이 어머니의 기억을 억제하기 시작했을 때, 버스 안에서 춤을 추는 수많은 어머니들은 남성의 세계를 어떤 기억과 함께 헤쳐왔을지 생각하게 된다.

\begin{thebibliography}{9}
  \bibitem[김혜남, 2000]{khn2000} 김혜남, \snm{영화 `박하사탕'에 대한 정신분석적 접근}, \bnm{정신분석} 제11권 제1호, 한국정신분석학회, 2000.
  \bibitem[최치원, 2011]{chw2011} 최치원, \snm{\oilpal 민주화운동에 대한 하나의 실존적 해석: `어두운 시대'에 감추어진 `악의 평범성' 문제}, \bnm{민주주의와 인권} 제11권 제3호, 전남대학교 5.18연구소, 2011.
  \bibitem[세라 블래퍼 허디, 2010]{sbh2010} 세라 블래퍼 허디, ``어머니의 탄생: 모성, 여성, 그리고 가족의 기원과 진화'', 사이언스북스, 황희선 역, 2010.
\end{thebibliography}
