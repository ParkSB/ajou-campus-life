\maketitle

% 서론
나는 영화 <진주 귀걸이를 한 소녀>(2003)의 내용이 실제 역사에 비춰봤을 때 얼마나 타당한지 당대 네덜란드의 사회적, 문화적 배경에 입각하여 분석하고자 한다. 영화는 그림 <진주 귀고리를 한 소녀>(1665)의 모델이 베르메르의 집에서 일하던 하녀였다는 가설을 제시한다. 그런데 실제로는 <진주 귀고리를 한 소녀>의 모델이 누구였는지, 그 모델이 베르메르와는 어떤 관계였는지, 심지어 그림의 소녀가 실존인물은 맞는지까지 밝혀진 바가 없다. 그래서 <진주 귀고리를 한 소녀>는 감상자로 하여금 다양한 맥락과 배경을 상상하게 만든다. 그림의 모델이 하녀였다는 추측도 수많은 가능성 중 하나로서, 그 추측이 어떤 맥락으로부터 도출되었는지, 또 그 추측이 충분한 설득력을 갖췄는지 논하는 과정에 의미가 있을 것이라고 생각한다.

% 당시 정치/사회/문화 배경.
<진주 귀고리를 한 소녀>가 그려진 17세기는 네덜란드 역사에서 황금시대로 평가 받는다. 독립 전쟁으로 스페인으로부터 독립을 쟁취한 유럽 북부의 동군연합은 네덜란드 공화국을 설립하고 정치적으로 안정을 찾는다. 이후 네덜란드는 동인도회사를 필두로 세계적인 무역국의 지위를 획득하며 경제적으로도 풍요를 누렸다. 또한 네덜란드는 16세기말 종교개혁으로 일찍이 개신교를 받아들였으며, 이에 따라 가톨릭 국가의 부유한 상인과 유능한 예술가들이 탄압을 피해 유럽 각지에서 네덜란드로 유입되었다. 이렇게 상업이 융성한 네덜란드에서는 부르주아 계급이 크게 성장했고, 최초의 근대적 미술시장이 형성되며 상품화된 미술품이 상인과 수집가들에 의해 거래되기도 하였다\footnote{조명계, \snm{네덜란드 미술시장 생성과정 연구}, \bnm{문화예술경영학연구} 제4권 2호, 한국문화예술경영학회, 2011, 88-89쪽.}. 따라서 자연스럽게 신이나 왕, 귀족을 묘사한 그림보다는 후원자의 초상화나 상인으로부터 의뢰를 받은 그림이 주로 그려졌다. 당대 최고의 화가로 평가받던 렘브란트 역시 초상화를 의뢰받아 그리며 명성을 쌓았다.

% 베르메르 개인 생애.
가톨릭 신자였던 베르메르는 초기에 종교적, 신화적 내용이 담긴 그림을 그렸지만, 나중에는 평범한 사람들의 일상을 묘사한 풍속화를 주로 그렸다. 베르메르는 서민층(low-middle class) 가정에서 자랐고, 비교적 부유한 집안의 딸이었던 카타리나와 결혼해 장모의 집에서 생활했다\footnote{Jonathan Janson, ``Vermeer's Life and Art'', Essential Vermeer, 2020, <www.essentialvermeer.com/vermeer's\_life.html>, (접근일자: 2023. 4. 9).}. 베르메르는 초상화 화가로서 인정받고 그림을 고가에 판매하기도 했지만, 작품을 완성하기 위해서는 오랜 시간을 들여야했기 때문에 계속해서 경제적 어려움을 겪을 수 밖에 없었다. <진주 귀고리를 한 소녀>를 완성하고 얼마 뒤에 네덜란드 미술 시장이 쇠퇴하기 시작했음을 돌아보면, 베르메르는 말년에도 어려운 시간을 보냈을 것으로 보인다. 베르메르가 세상을 떠난 뒤 아내 카타리나가 그의 작품을 하나씩 판매했다는 사실은 그런 생활고에 대한 방증일지도 모른다. 베르메르의 성장 배경과 당시 그가 처했던 상황을 생각해보면 베르메르는 하녀에게 계급적 이질감 내지는 혐오를 느끼기 보다는, 영화에서 묘사하듯 하녀를 동등한 인격체로 바라봤을 가능성도 충분히 있다.

% 결론
여기까지 그림 <진주 귀고리를 한 소녀>의 모델이 하녀였다는 영화 <진주 귀걸이를 한 소녀>의 가설이 타당한지 논하기 위해 17세기 네덜란드의 사회적, 문화적 배경과 당시 베르나르의 상황을 살펴보았다. 당시에는 화가가 풍속화를 그리는 것이 일반적이었고, 베르나르가 서민층 가정에서 자랐다는 배경을 비춰봤을 때, 그림의 모델이 하녀라는 추측도 일견 그럴듯해 보인다. 그림에 고가의 물감인 울트라마린을 사용했다는 점에서 베르메르와 그림의 모델이 각별한 사이였을 것이라는 추측도 있는데, 영화에서 베르메르와 하녀가 미묘한 감정을 나누는 것으로 묘사되는 데는 이러한 추측이 작용한 것으로 보인다. 다만 여기에는 역사적 근거가 있기 보다는, 예술적 상상에 가까운 것이므로 이 글에서 추가적으로 분석할 필요는 없다고 판단했다. <진주 귀고리를 한 소녀>는 오늘날에도 여전히 높은 평가를 받는 작품이다. 섬세한 빛 표현과 역동적인 구도, 선명한 색채 뿐만 아니라, 다양한 해석이 가능하다는 점 역시 이 작품의 매력을 더욱 높이는 요소라고 생각한다.
