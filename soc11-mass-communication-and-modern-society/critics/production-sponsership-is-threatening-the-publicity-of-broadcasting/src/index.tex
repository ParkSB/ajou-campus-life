\maketitle

또래에 비해 건강하다는 중년이 자신의 건강 관리 비결을 `알티지 오메가3'라고 소개하자 스튜디오의 패널들이 연신 감탄을 내뱉는다. 잠시 후 패널로 자리한 가정의학과 전문의가 알티지 오메가3는 일반 오메가3보다 훨씬 흡수율이 높다는 설명을 이어간다. 같은 시각, 바로 옆 홈쇼핑 채널에서는 알티지 오메가3 상품을 단독 판매하고 있다. 홈쇼핑 판매 프로그램은 그저 우연히 바로 옆 채널의 건강 정보 프로그램에서 소개하는 알티지 오메가3를 판매하고 있던 것일까?

오늘날 텔레비전 방송에 제작협찬은 필수적인 요소가 되었다. 방송법에 따르면 제작협찬은 방송 제작자가 제3자로부터 물품이나 용역 등을 제공받는 것을 말한다. 제작자는 프로그램 제작에 필요한 각종 비용을 절약할 수 있고, 협찬주는 협찬 상품이나 브랜드를 홍보할 수 있다는 장점이 있다. 그런데 최근에는 협찬의 광고 효과를 보장하기 위해 방송 프로그램이 도를 넘는 수준으로 협찬 상품을 강조하고 있으며, 심지어 홈쇼핑 채널과 프로그램 편성을 연계하여 협찬 상품 구매를 유도하는 `광고 전략'이 성행하고 있다.

이처럼 방송 프로그램에 무분별하게 스며들어 노출되는 협찬 상품 및 브랜드 등은 시청자의 합리적인 의사 결정을 방해하고, 정보의 비대칭성을 이용해 시청자를 기만한다는 점에서 명백히 방송의 공공성을 침해하기 때문에 비판적으로 살펴볼 필요가 있다. 이 글에서는 방송의 이념 중 공공성과 공익성을 근거로, 앞서 사례로 든 건강 정보 프로그램을 비롯해 드라마, 예능 등 텔레비전 방송에서 적극적으로 활용하는 제작협찬에 대해 비평하고자 한다.

\section*{방송의 공공성과 공익성}

방송은 공적 커뮤니케이션 매체로서 공공성과 공익성을 지닌다. 공영방송이 아니더라도 방송은 본질적으로 불특정 다수의 공중을 대상으로 하며, 한정된 방송 주파수를 이용한다는 특성으로 인해 공동의 가치를 추구할 의무가 있다. 방송의 공공성과 공익성은 방송법 제5조(방송의 공적 책임)와 제6조(방송의 공정성과 공익성)에 명시되어 있으며, 이러한 맥락에서 제73조(방송광고등) 제1항은 방송사업자가 방송광고와 방송프로그램이 혼동되지 않도록 명확하게 구분해야 한다는 점을 적시하고 있다.

제작협찬은 방송통신위원회규칙 제56호(협찬고지 등에 관한 규칙)에 ``방송프로그램제작자가 방송제작에 관여하지 않는 자로부터 방송프로그램의 제작에 직 $\cdot$ 간접적으로 필요한 경비 $\cdot$ 물품 $\cdot$ 용역 $\cdot$ 인력 또는 장소 등을 제공받는 것''이라고 정의되어 있다. 제작협찬은 간접광고와 비슷해 보이지만, 결정적으로 간접광고와 달리 상표를 노출할 수 없다. 즉, 상품을 협찬하는 경우 상품이 방송에 노출되기는 하지만 상표는 가려야 하는 것이다. 제작협찬은 광고가 아니므로 방송법 제73조에 해당하지 않으며, 제74조(협찬고지)에도 ``협찬고지를 할 수 있다'' 정도만 명시되어 있다. 이처럼 현재 법률은 협찬고지에 대해서만, 그것도 권고 형태로 규정하고 있으며, 제작협찬은 단지 행정규칙인 방송통신위원회규칙으로만 규제받고 있어 설령 제작협찬이 광고와 유사하게 사용된다고 하더라도 이를 제재할 법적 근거는 미비한 상태다.

한편 2010년 방송법 및 방송법 시행령 개정으로 제도권에 들어온 간접광고는 방송법에 ``방송프로그램 안에서 상품, 상표, 회사나 서비스의 명칭이나 로고 등을 노출하는 형태의 광고''로 정의되어 있다. 간접광고에 관한 논의는 비교적 오랜 기간 이뤄졌기 때문에 제작협찬에 비해 구체적인 법적 규제가 마련되어 있다. 또한 간접광고는 한국방송광고진흥공사나 민영 미디어렙\footnote{미디어렙(Media Representative, 광고 매체 판매 대행사): 방송사에 광고를 위탁 판매하고 광고주로부터 일정액의 수수료를 받는 회사.}을 거쳐 약정되기 때문에 어느정도 투명성이 확보되지만, 제작협찬은 방송 제작 주체와 협찬주 사이 직접 거래로 약정\cite{hkh2015}되기 때문에 보다 음성적이라고 할 수 있다.

법적으로는 간접광고와 제작협찬이 구분되어 있지만, 시청자의 관점으로는 이 둘을 구분하기가 쉽지 않다. 따라서 문제의식의 초점을 제작협찬 그 자체보다는 시청자가 프로그램 중 노출되는 상품이 실제로는 광고 효과를 의도하고 노출되는 것임을 인지할 수 없다는 점에 맞추는 것이 적절할 수도 있다. 다만 이 비평에서는 논의의 엄밀함을 위해 제작협찬으로 그 범위를 분명히 하겠다.

\section*{광고 전략으로서의 제작협찬}

% 현황, 범위 설정.

제작협찬에서 문제가 되는 지점은 방송 프로그램을 보는 시청자가 프로그램 내에 노출되는 상품이 광고를 목적으로 노출되는 것인지, 단순 협찬으로 노출되는 것인지, 아니면 광고나 협찬과 무관하게 노출되는 것인지 쉽게 인지할 수 없는 경우다. 한국방송광고진흥공사에 따르면\cite{ndr2013} 방송사 및 외주제작사는 수입을 온전히 얻기 위해 광고주에게 제작협찬의 광고 효과를 보장하며 관행적으로 간접광고가 아닌 제작협찬을 유도한다. 이때 상표 노출 없이 광고 효과를 보장하기 위해 방송 프로그램은 다양한 형태를 취하게 된다.

최근 가장 논란이 되는 형태는 `연계편성'이다. 연계편성은 한 프로그램과 다른 프로그램을 말 그대로 연계해 방송하는 방식의 편성이며, 이를 가장 적극적으로 활용하는 곳은 지상파 및 종합편성채널의 건강 정보 프로그램과 홈쇼핑 채널이다. 가령 건강 정보 프로그램에서 전문가 패널을 섭외해 특정 건강식품이 건강에 좋다는 방송을 하는 동시에, 또는 방송이 끝난 직후 바로 옆 홈쇼핑 채널에서 동일한 건강식품을 판매하는 식이다. 방송통신위원회에서 발표한 2022년 건강 정보 프로그램과 홈쇼핑 간 연계편성 점검결과\cite{kcc2022}에 따르면 지상파 및 종합편성채널 51개 프로그램이 7월 한 달 동안 총 753회 홈쇼핑 채널과 연계편성을 했다. 주요 협찬 상품은 단백질, 유산균과 같은 건강식품이었다. 충격적인 점은 지상파 방송사도 연계편성을 했다는 것이다. MBC는 3개 프로그램에서 총 46회, SBS는 4개 프로그램에서 총 25회 연계편성을 했다. 준공영방송의 영역에 있는 MBC가 제작협찬으로 홈쇼핑과 연계편성을 했다는 사실은 방송의 공공성이 그만큼 크게 침해받고 있다는 방증이다.

협찬 상품은 간접광고 상품과 달리 상표를 노출할 수 없지만, 이처럼 별도로 연계편성을 하면 상표를 노출한 것과 비슷한 광고 효과를 볼 수 있다. 건강 정보 프로그램은 특정 상품을 마치 공익적인 목적으로 소개하듯 연출하지만, 실제로는 대가를 받고 상품을 노출하는 광고에 가까운 것이다. 홈쇼핑 채널 번호가 지상파나 종합편성채널, 또는 인기 채널 사이사이에 할당되어 있는데다가, 프로그램의 주 시청 연령층이 텔레비전 방송 시청률과 신뢰도가 높고\cite{kpf2021} 홈쇼핑에 적극적인 장 $\cdot$ 노년층이라는 점에서 협찬 상품의 광고 효과가 상당할 것이라는 예상은 어렵지 않다.

뉴스타파는 더 나아가 제작협찬의 문제를 고발하기 위해 SBS의 자회사 채널 SBS비즈의 생활 정보 프로그램에 협찬금을 지불하고 가상의 체리 상품을 홍보하는 기획을 했다\cite{newstapa2021}. SBS비즈는 기본적인 검증도 없이 협찬금 660만 원을 받고 가상의 상품과 가상의 전문가를 방송에 노출했다. 협찬주보다 더 적극적으로 상품을 어필하고 광고 효과를 극대화하는 방향으로 방송을 연출하는 모습은 방송사보다는 광고기획사에 가까운 모습이다. 심지어 시사 $\cdot$ 보도 프로그램도 이 문제에서 자유롭지 않다. 과거 MBN의 시사 프로그램 <경제포커스>는 공기업의 자원외교 실패에 대한 내용을 보도하면서 제작협찬을 받은 한국전력의 성공 사례를 강조했다. 그러나 이에 대한 제재는 행정규칙뿐이라 방송통신위원회가 부과하는 소액의 과징금으로 그쳤다\cite{mediatoday2015}.

특히 종합편성채널은 과할 정도로 생활, 건강 정보 프로그램을 편성하고 제작협찬을 유치하며 수익을 극대화하고 있다. JTBC, 채널A, MBN은 모두 2021년 협찬받은 상품의 효과를 다룰 때 협찬 고지를 3회 이상하지 않아 방송통신위원회의 시정명령을 받았다. TV조선은 연계편성을 하면서도 홈쇼핑 채널에 판매하는 상품과 프로그램에 노출하는 상품을 불일치시키는 편법으로 협찬고지 의무를 회피했다\cite{mediaus2021}.

\section*{`뒷광고' 논란과 협찬의 법제화}

% 문제 정의, 비판.

시청자 입장에서 방송이 제작협찬의 광고 효과를 보장하기 위해 갖은 방법을 동원한다는 사실은 단순히 돈을 내면 협찬 상품을 홍보할 수 있다는 인식에서 그치지 않는다. 결국 돈만 내면 사실이 아닌 것을 사실처럼 방송할 수 있다는 예상으로 이어진다. 이것은 방송과 언론 전체에 대한 신뢰성을 위협한다.

공공성, 공익성의 측면에서 방송의 광고와 수익 추구는 이념적, 윤리적 문제이기도 하지만, 한편으로는 방송 산업의 재원 마련과 직결된 현실적인 문제라는 지적도 있다. 그러나 텔레비전 방송사 및 제작사의 경영 악화를 타개하겠다는 이유로 방송 스스로 공공성과 신뢰성을 포기한다면 결국 유튜브를 비롯한 인터넷 미디어에 대응할 수 있는 유일한 경쟁력을 완전히 잃어버리고 말 것이다. 이 또한 현실적인 문제임을 잊어서는 안 된다.

제작협찬은 시장과 언론의 자유 영역이므로 자율에 맡겨야 한다는 주장도 있다. 그러나 시청자가 상품의 협찬 사실을 인지하지 못하도록 해 광고 효과를 유도하는 것은 오히려 소비자의 합리적인 의사 결정을 방해함으로써 시장을 교란하는 행위다. 또한 제작협찬을 전적으로 언론의 자유에 맡겨야 한다는 주장은 방송의 공공성과 공익성을 망각한 주장이다. 언론의 자유는 정치 및 자본 등 권력으로부터의 독립성을 보장해 민주주의를 보호하기 위한 원칙이지, 기업이 방송을 무한한 광고 도구로 사용할 수 있게 방치해야 한다는 원칙이 아니다.

재작년 유튜브의 인터넷 방송인들이 상품의 협찬 사실을 알리지 않고 콘텐츠를 제작하여 큰 논란이 됐다. 이른바 `뒷광고 사건'으로 불린 이 논란은 인터넷 방송인 여럿이 방송을 중단해야 했을 정도로 여론의 질타를 받았다. 당시 기성 언론은 인터넷 방송을 강하게 비판했지만, 자신을 돌아보지는 않은 것 같다. 이제 협찬 상품을 소개하는 유튜브 영상에는 `유료 광고 포함'이라는 문구가 노출되는 것이 자연스러운 일이 되었다. 반면 텔레비전 방송 프로그램은 방송통신위원회규칙에 따라 세 번만 협찬 고지를 하면 된다. 그마저도 프로그램 시작 전, 끝난 후 고지하는 경우가 일반적이기 때문에 시청자가 프로그램에 등장한 상품이 협찬 상품임을 인지하는 것은 매우 어려울 수밖에 없다. 협찬 사실을 시청자에게 투명하게 제공하기 위해서는 협찬 상품이 노출되는 그 시점에 협찬고지를 해야 한다.

2020년 10월 문재인 정부에서는 제작협찬의 투명성을 제고하기 위한 방송법 개정안을 국무회의에서 의결하고 국회에 제출했지만, 2년이 지난 지금도 여전히 국회에서 계류 중이다. 방송법 개정안의 핵심은 제작협찬에 대한 기준을 방송법으로 정의하는 것이다. 여기에는 시사 $\cdot$ 보도 등 특정 분류의 프로그램에 대한 협찬을 금지하고, 상품의 효능이나 효과를 다루는 경우 필수적으로 협찬을 고지하도록 하는 내용이 포함되었다. 현재 방송통신위원회가 행정규칙으로 제재 권한을 행사할 수 있지만, 이는 사안별 조치일 뿐이기 때문에 방송 전반에서 협찬을 정상화하기 위해서는 법률로 규제할 필요가 있다\cite{ksj2020}. 방송법 개정안은 현재 행정규칙으로만 관리하는 협찬 관련 규제를 법제화하여 규제의 실효성을 확보하는 데 큰 의미가 있다.

현재로서는 제작협찬을 관리 $\cdot$ 감독할 방법이 마땅치 않다. 편법, 위법 협찬이 성행하는 원인 중 하나로 협찬의 운영이 투명하지 않다는 점이 지적되기도 한다\cite{jyw2016}. 방송법 개정안에는 방송사가 협찬 관련 자료를 일정 기간 보관하고, 요청에 따라 제출하도록 하는 내용도 포함되어 있다. 실제로 방송통신위원회가 협찬 및 연계편성의 증거를 찾을 수 없다는 이유로 방송 프로그램을 제재하지 못하는 사례가 많기 때문에 제작협찬의 투명성을 보장하는 조항은 특히 필요해 보인다. 시청자를 기만하는 방식의 광고성 협찬으로부터 방송의 공공성을 지킬 법적 근거를 마련하려면 방송이 더 신뢰를 잃기 전에 개정안을 입법해야 한다.

\section*{자본의 방송에서 공공의 방송으로}

여기까지 방송의 이념 중 공공성과 공익성을 근거로 제작협찬의 기만적 광고 효과와 그 실상을 비평했다. 제작협찬은 사실상 간접광고의 일종에 가까운 형태로 활용되고 있지만, 법적 제재가 가능한 간접광고와 달리 행정규칙인 방송통신위원회규칙으로만 규제받고 있어 광고성 협찬으로부터 공공성을 보호하기에는 한계가 있다.

이어서 실제 사례를 검토함으로써 제작협찬은 상품의 강조 연출, 연계편성, 협찬금 지급을 통한 방송 구성 등 다양한 방식으로 광고 효과를 보장받고 있음을 확인했다. 시청자는 프로그램에 노출되는 상품의 협찬 여부를 쉽게 인지하기 어렵기 때문에 이러한 방식의 제작협찬은 소비자의 합리적인 의사 결정을 방해한다. 또한 자본이 방송을 강하게 통제하고 있다는 사실은 언론에 대한 신뢰를 저하하며, 방송의 공공성과 공익성, 투명성을 심각하게 훼손한다.

마지막으로 텔레비전 방송의 제작협찬을 인터넷 방송의 `뒷광고 사건'과 대조하며 협찬 상품이 등장하는 시점에 협찬고지를 해야 함을 밝혔다. 또한 제작협찬을 법률로 다루고 감독하려면 2020년 정부에서 의결한 방송법 개정안을 속히 입법해야 한다고 주장했다. 이 비평을 통해 말하고자 하는 것은 `방송이 광고를 하면 안 된다'라는 주장이 아니다. 적어도 지금처럼 시청자를 기만하는 방식으로 광고를 해서는 안 된다는 것이 핵심이다. 방송의 공공성과 공익성을 전제하는 시청자에게 방송에 노출되는 상품이나 브랜드는 방송에 노출된다는 사실 자체만으로도 객관성과 공신력을 보장받는다. 방송 스스로 그러한 사실에 공적인 책임감을 가질 필요가 있다. 훗날 역사에 텔레비전 방송의 마지막 순간에도 방송은 기만적 광고로 점철되어 있었다고 평가받지 않길 바라며 비평을 마친다.

\begin{thebibliography}{9}
  \bibitem[김서중, 2020]{ksj2020} 김서중, \snm{광고 정책과 시민의 권익}, \bnm{황해문화} 제109호, 새얼문화재단, 2020, 275쪽.
  \bibitem[정연우, 2016]{jyw2016} 정연우, \snm{방송 협찬 제도 개선 방안 연구}, \bnm{정치커뮤니케이션연구} 제40호, 한국정치커뮤니케이션학회, 2016, 126쪽.
  \bibitem[한규훈 $\cdot$ 문장호, 2015]{hkh2015} 한규훈 $\cdot$ 문장호, \snm{국내 간접광고 규제의 개선방향에 관한 고찰}, \bnm{광고연구} 제104호, 한국광고홍보학회, 2015, 113쪽.
  \bibitem[노동렬 외, 2013]{ndr2013} 노동렬 외 3명, ``간접광고 도입 등에 따른 협찬제도의 효과적 규제방안 연구'', 한국방송광고진흥공사, 2013, 95쪽.
  \bibitem[방송통신위원회, 2022]{kcc2022} ``방통위, 방송사의 \textquotesingle22년도 연계편성 점검결과 발표'', <방송통신위원회>, 2022.
  \bibitem[한국언론진흥재단, 2021]{kpf2021} ``2021 언론수용자 조사'', <한국언론진흥재단>, 2021.
  \bibitem[뉴스타파, 2021]{newstapa2021} ``SBS부터 JTBC까지 `사기 전기차' 홍보...언론은 공범인가? 주범인가?'', <뉴스타파>, 2021, <https://newstapa.org/article/ULW9Q>, (2022. 10. 24).
  \bibitem[미디어오늘, 2015]{mediatoday2015} ``MBN 변칙 광고영업, 고작 벌금 500만원'', <미디어오늘>, 2015, <https://www.mediatoday.co.kr/news/articleView.html?idxno=125100>, (2022. 10. 24).
  \bibitem[미디어스, 2021]{mediaus2021} ``TV조선이 `홈쇼핑 연계편성 고지' 재승인 조건을 회피한 방법'', <미디어스>, 2021, <https://www.mediaus.co.kr/news/articleView.html?idxno=228686>, (2022. 10. 24).
\end{thebibliography}
